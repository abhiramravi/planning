\documentclass{article}
\usepackage{fullpage}
\usepackage[colorlinks=true,linkcolor=black,urlcolor=black]{hyperref}

\title{
{\large 15-887: Planning, Execution, and Learning}\\
Project Proposal}
\author{Carlo Angiuli \and Ben Blum \and Michael Sullivan}
\date{November 5, 2012}

\begin{document}
\maketitle

\section{Summary}

In programming language theory, \emph{substructural operational semantics}
(SSOS) is a new technique for modularly specifying programming language
semantics using linear logic.%
\footnote{For example, in Pfenning, F.,
\textit{Substructural Operational Semantics as Ordered Logic Programming}.
(\url{http://dx.doi.org/10.1109/LICS.2009.8})}

\begin{enumerate}
\item linear logic proof search is planning
\item essentially, using planning as model checking
\item show safety of programs
\item what's good about modularity?
\end{enumerate}

\section{Proposal}

\begin{enumerate}
\item the problem
\item why is it a planning problem?
\item our approach
\item evaluation
\end{enumerate}

\end{document}
