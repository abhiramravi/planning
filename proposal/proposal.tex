\documentclass{article}
\usepackage{fullpage}
\usepackage[colorlinks=true,linkcolor=black,urlcolor=black]{hyperref}

\title{
{\large 15-887: Planning, Execution, and Learning}\\
Project Proposal}
\author{Carlo Angiuli \and Ben Blum \and Michael Sullivan}
\date{November 5, 2012}

\begin{document}
\maketitle

\section{Summary}

In programming language theory, \emph{substructural operational semantics}
(SSOS) is a new technique for modularly specifying programming language
semantics using linear logic.%
\footnote{For example, in Pfenning, F.,
\textit{Substructural Operational Semantics as Ordered Logic Programming}.
(\url{http://dx.doi.org/10.1109/LICS.2009.8})}
Traditionally, programming languages can be specified by abstract machines which
transition between program states. SSOS encodes program states as sets or
sequences of linear propositions, and abstract machine transitions in terms of
linear implications which consume and produce those propositions.

Then program executions correspond exactly to proofs in linear logic, so 


if
ill-formed states are unprovable, then we know a program to be safe.


\begin{enumerate}
\item linear logic proof search is planning
\item essentially, using planning as model checking
\item show safety of programs
\item what's good about modularity?
\end{enumerate}

\section{Proposal}

It has been observed many times that planning
corresponds closely to linear logic proof search.


operators which abstract over domains
respect the frame rule

\begin{enumerate}
\item the problem
\item why is it a planning problem?
\item our approach
\item evaluation
\end{enumerate}

\end{document}
