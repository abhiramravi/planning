\documentclass{article}
\usepackage{fullpage}
\usepackage{multicol}
\usepackage[colorlinks=true,citecolor=black,linkcolor=black,urlcolor=black]{hyperref}

\title{
{\large 15-887: Planning, Execution, and Learning}\\
{\bf Program Verification as Planning}}
\author{Carlo Angiuli \and Ben Blum \and Michael Sullivan}
\date{December 5, 2012}

\begin{document}
\maketitle

\section{Introduction}

Program verification is the problem of showing that a computer program correctly
implements a specification of its intended behavior, or finding a
counterexample. These verifications can be either \emph{formal} or
\emph{informal}. In formal verification, one mathematically defines both the
specification and the program's own semantics, using techniques such as
substructural operational semantics~\cite{rob}, and proves that these coincide.
Informal verifications are prone to false positives or false negatives---but are
more immediately applicable to a broader range of programs. Informal analyses
are often used in systems research to reason about ill-specified code ``in the
real world'', which is especially useful for finding concurrency
errors~\cite{ben}.

{\em Planners} take a domain definition and initial state, and compute a
sequence of actions for achieving a goal; a planner is {\em complete} when its
failure to achieve a goal guarantees that no sequence of actions can achieve
that goal.
We investigate the correspondence between planning and program verification: the domain is the programming language (or the abstract machine), the initial state is the program being verified, the goal is the specification, and the plans are execution traces.
If a complete planner cannot plan for an invalid state, this constitutes a
formal verification that a program is safe (assuming that the planner is
correct).

We investigate how planners can be used for program verification in two particular domains:
first, for reasoning about functional and logic programs using an encoding
inspired by substructural operational semantics,
and second, for reasoning about concurrent computation with a focus on synchronisation algorithms.
Our results are mixed.
Of the two modern planners we used, none were able to efficiently execute lambda calculus programs.
On the other hand, we found planners were quite efficient at checking concurrent programs against their specifications,
and could even be used to reason about nonterminating programs.
Inspired by this, we also provide a compiler for automatically generating PDDL
domains for concurrent programs from a more natural representation.
We discuss why modern planners proved inadequate for reasoning about functional programming, and also discuss the implications for concurrency verification.

\section{Functional and logic programming}

% Functional and logic programming

\subsection{Plus domain}

Logic programming (as in, e.g., Prolog or Twelf) is a style of declarative
programming in which computations are expressed as predicates which hold exactly
when the conjunction of other predicates hold; executing a program amounts to
searching for a proof of the desired predicate.

A simple example is addition of natural numbers. We represent natural numbers as
either zero or the successor of a natural number ($\texttt{z}, \texttt{s(z)},
\texttt{s(s(z))}$, etc.). Addition is expressed as a predicate
\texttt{plus(M,N,P)}, which holds exactly when $M+N=P$, and could be implemented
in Prolog as:
\begin{verbatim}
plus(z,N,N).
plus(s(M),N,s(P)) :- plus(M,N,P).
\end{verbatim}
The first line says that $\texttt{z}+N=N$, for any $N$, and the second line says
that $\texttt{s(}M\texttt{)}+N=\texttt{s(}P\texttt{)}$ so long as $M+N=P$. This
is remarkably similar to a planning problem, except for the crucial difficulty
that Prolog natively supports producing and consuming nested data like
\texttt{s(s(z))} and instantiates rules via unification of this nested data.

To represent nested data types in PDDL, we were forced to represent them
explicitly via pointer graphs of constructors. \texttt{z} is represented by any
pointer object $\texttt{p}n$ accompanied by the fact \texttt{(Z p$n$)};
similarly, \texttt{s(z)} is represented by an object $\texttt{p}n$ accompanied
by the facts \texttt{(S p$n$ p$m$)} and \texttt{(Z p$m$)}. Thus the first rule
for addition can simply be encoded as an action with precondition \texttt{(Z
?n1)} and effect \texttt{(Plus ?n1 ?n2 ?n2)}.

The second rule, however, requires creating a new \texttt{S} constructor to wrap
around the $P$ result of the recursive call. It is essential that this
constructor is ``allocated'' at a new pointer, to avoid conflicting with any
other constructor in existence. So we keep a fact \texttt{(ptr-next p$n$)} to
store the next currently unused pointer, and increment $n$ every time we use
this pointer (via a set of hardcoded facts \texttt{(ptr-succ p$n$ p$(n+1)$)}).
The implementation looks like this:
{\small
\begin{verbatim}
  (:action Plus-S
    :parameters (?n1 ?n2 ?n1p ?fresh ?freshp ?ansp - ptr)
    :precondition (and
      (Plus ?n1p ?n2 ?ansp)
      (S ?n1 ?n1p)
      (ptr-next ?fresh)
      (ptr-succ ?fresh ?freshp))
    :effect (and
      (not (Plus ?n1p ?n2 ?ansp))
      (not (ptr-next ?fresh))
      (ptr-next ?freshp)
      (Plus ?n1 ?n2 ?fresh)
      (S ?fresh ?ansp)))
\end{verbatim}}
If we strip away all the pointer manipulation, this simply states that if
\texttt{(Plus ?n1p ?n2 ?ansp)}, then \texttt{(Plus (S ?n1p) ?n2 (S ?ansp))}.

Our results here were mixed. SGPlan was immediately able to solve problems like
$2+2$ and $4+2$, but as soon as we provisioned for pointers beyond \texttt{p14}
(with \texttt{ptr-succ} facts), SGPlan would mysteriously exit silently during
the preprocessing phase, regardless of the goal. We couldn't diagnose this
error, but suspect that the planner ran out of memory while computing
constraints between all the \texttt{ptr-succ} facts. As a result, sadly, SGPlan
cannot compute $n+m$ for $2n+m>12$, since this would require over $15$ pointers.

\subsection{Lambda calculus domain}

Although the \texttt{plus} domain above quickly exhausted the planner's
resources, we were at least able to translate a traditional logic program into
PDDL, so we next attempted to implement the lambda calculus in a similar logic
programming fashion. The workhorse of lambda calculus is substitution---when a
function is applied to an argument, we substitute this argument for all
occurrences of the formal parameter in the body of the function. (As in the
\texttt{plus} domain, we represent lambda terms by pointer graphs.)

We attempted to implement substitution by a predicate \texttt{(subst e1 x e2 e)}
which represents the fact that \texttt{e1} substituted for \texttt{x} in
\texttt{e2} is \texttt{e}, and added these rules to an implementation of natural
numbers.

Unfortunately, both MIPS and SGPlan throw an out-of-memory error while
preprocessing these domains, even when the problem is simply to evaluate the
number \texttt{(S Z)}, which evaluates to itself. Given the limited ability of
the planners to perform addition, it is not surprising that substitution, a
complex pointer manipulation problem, was not preprocessable whatsoever.

\subsection{SKI combinator domain}

Our final attempt at logic programming was to encode the SKI combinator
calculus, a formulation of lambda calculus which has no variables whatsoever,
but only three constants $S,K,I$ (and applications thereof) satisfying
particular reduction rules. This has the benefit of sidestepping substitution
entirely, and is the simplest Turing-complete functional language.

Here, we implemented an abstract stack machine following the methodology of
substructural operational semantics (SSOS), a new technique of efficiently
representing abstract machines as transition systems in ordered logic. In this
context, we can think of ordered logic as a planning framework (based on linear
logic%
\footnote{The similarity between linear logic proof search and classical
planning has been observed multiple times in the past, as in \cite{Dixon}.}%
) in which facts are stored in a sequence, and operators consume and
replace sets of contiguous facts in this sequence. This sequencing allows us to
represent stack frames without explicitly keeping track of their order.



Of course, planners also do not support ordered logic, so we encode the stack
frames as a linked list of facts, each of which takes a number of pointers to
combinator expressions, as in the previous domains. The PDDL implementation has
13 operators; we evaluate a combinator expression $E$ by starting with the fact
\texttt{(eval p0 $E$ pn)} (where p0 and pn are the stack frame pointers) and
planning for \texttt{(retn p0 $E$ pn)}, a fact whose existence indicates that
the stack is empty except for that return frame.

Unsurprisingly, SGPlan also runs out of memory while planning on this domain,
even for the simplest possible combinator term, $I$ (the identity combinator).
MIPS attempts to fully instantiate the domain, and cannot even complete
preprocessing. (As a side note, we had to recompile SGPlan to load this domain,
because the \texttt{Retn-S2} instruction has 11 parameters, above the default
maximum of 9.)

\subsection{Analysis}

We were successfully able to encode a variety of logic programs as classical
planning domains, ranging from addition in Prolog to an SSOS encoding of
combinator calculus in ordered logic. This supports our original hypothesis that
logic programming can be viewed as a particular kind of planning problem, and
can be encoded (albeit awkwardly) in PDDL. The biggest hurdle here is that logic
programming is fundamentally based on unification of nested data, and PDDL
requires that we instead express this data using pointers.

The difficulty is that planners attempt to statically analyze the domain and
problem, computing mutual exclusion, subdividing the problem, etc., but these
domains simply cannot be handled in this fashion, because (1) the operators have
many parameters all of the same type, due to the pointer conversion, and (2)
there is precisely one plan to evaluate any addition/lambda term/combinator
term, and each operator in this plan is uniquely determined by the structure of,
respectively, the first summand and the control stack. We tried to alleviate (1)
by using the \texttt{:typing} extension of PDDL to distinguish between, e.g.,
stack frame pointers and expression pointers, to limit the number of
instantiations, but the planners still ran out of memory.

The deterministic nature of the operational semantics applies only to
``well-formed'' initial states, and is a non-trivial fact. It is in fact very
easy to find this unique plan by starting at the initial state, determining the
unique operator and instantiation which can be applied at each state, and
repeating until the goal is satisfied; the planners' preprocessing consumes all
available memory but does not actually simplify the problem whatsoever. 




\section{Concurrent imperative programming}

\subsection{Threads domain}

We also encoded a simple assembly-like abstract machine with a fork/join primitive as a planning domain, which we call the ``threads'' domain.
Its predicates are used to indicate values of data and to indicate a program's instruction sequence, and
its actions represent the language's evaluation rules. For example, \texttt{value ?name ?value} indicates the value of a variable with a given name. Each instruction is encoded as both a predicate (its representation in the instruction stream) and an action (which specifies its semantics).
%
\begin{itemize}
	\item \texttt{set ?me ?next ?name ?value} - Assigns a value to a variable. \texttt{me} is the label for this instruction (all instructions share this pattern), \texttt{next} is the label of the next instruction to evaluate after this one. The increment, decrement, load, atomic-exchange, and atomic-add instructions are similar.
	\item \texttt{branch ?me ?name ?iftrue ?iffalse} - Flow control. Instead of \texttt{next}, there are two next instructions, selected between depending whether the \texttt{value name n0} fact exists (i.e., \texttt{name == 0}).
	\item \texttt{exit ?me} - Terminates execution of the current thread (or program).
	\item \texttt{fork ?me ?next ?child1 ?child2} - Runs \texttt{child1} and \texttt{child2} to completion (both must \texttt{exit}), and then advances to \texttt{next}.
\end{itemize}
%
Finally, \texttt{eval ?instruction ?out} represents the program counter; its first argument is the label associated with the next instruction to be executed, and its second argument is the ``destination'' label, a special label used to identify which threads have terminated. There are as many \texttt{eval} tokens as currently-running threads.

The first set of problems in the threads domain demonstrates a simple data race between two threads---two interleaving threads attempting to increment a shared variable can nondeterministically produce different results. The initial state of the planning problem expresses the following program:
%
	\begin{center} \small
	\begin{tabular}{ll}
	\multicolumn{2}{c}{\texttt{x = 0;~~~~~~~~~~~~~~~~~~}} \\
	\multicolumn{2}{c}{\texttt{fork(thread1, thread2);}} \\
	& \\
	\texttt{thread1() \{} & \texttt{thread2() \{} \\
	\texttt{~~~~temp0 = x;\qquad} & \texttt{~~~~temp1 = x;} \\
	\texttt{~~~~temp0++;} & \texttt{~~~~temp1++;} \\
	\texttt{~~~~x = temp0;} & \texttt{~~~~x = temp1;} \\
	\texttt{\}} & \texttt{\}} \\
	\end{tabular}
	\end{center}
%
In this example, possible values for \texttt{x} at the end are 1 and 2. The problems with filenames starting in \texttt{prob1} demonstrate this problem; those with filenames starting in \texttt{prob2} and \texttt{prob3} are elaborations on it.

\subsection{Verifying synchronisation algorithms}

Next, we used planners to verify several different algorithms for {\em synchronisation} -- the problem of protecting designated ``critical sections'' of execution from unsafe concurrent access. Mutual exclusion algorithms are characterised by three properties: {\em mutual exclusion}, {\em bounded waiting}, and {\em progress}~\cite{de0u}.

\paragraph{Mutual Exclusion}
An algorithm that provides mutual exclusion does not allow multiple threads to be executing in the critical section simultaneously. To ensure that an algorithm provides mutual exclusion, we write programs of the following form, in which both threads modify a \texttt{num\_in\_section} counter to indicate when they're in the critical section:
%
\begin{center} \small
\begin{tabular}{ll}
\multicolumn{2}{c}{\texttt{num\_in\_section = 0;~~~~~~}} \\
\multicolumn{2}{c}{\texttt{fork(thread1, thread2);}} \\
& \\
\texttt{thread1() \{} & \texttt{thread2() \{} \\
\texttt{~~~~while (true) \{} & \texttt{~~~~while (true) \{} \\
\texttt{~~~~~~~~\em lock\_sequence();} & \texttt{~~~~~~~~\em lock\_sequence();} \\
\texttt{~~~~~~~~num\_in\_section++;} & \texttt{~~~~~~~~num\_in\_section++;} \\
\texttt{~~~~~~~~num\_in\_section--;} & \texttt{~~~~~~~~num\_in\_section--;} \\
\texttt{~~~~~~~~\em unlock\_sequence();\qquad} & \texttt{~~~~~~~~\em unlock\_sequence();} \\
\texttt{~~~~\}} & \texttt{~~~~\}} \\
\texttt{\}} & \texttt{\}} \\
\end{tabular}
\end{center}
%
We then set the goal statement to be \texttt{num\_in\_section == 2}. When a complete planner cannot plan for this fact, it guarantees that no execution interleaving exists in which both threads are in the section at once -- in other words, that mutual exclusion is satisfied.

We add the infinite loop in each thread's body to test the unlock sequence as well as the lock sequence, in case a broken algorithm requires multiple iterations before failing. One particular strength of planners shines here: their ability to deal with cyclic state spaces. In the blocks world, no planner worth its salt would get stuck evaluating the infinite plan ``pick up block A, stack A on B, pick up block A, \dots'', and likewise, here the planners easily cope with the infinite loops in each thread.

\paragraph{Bounded Waiting}
An algorithm that provides bounded waiting does not allow any thread to be ``starved'' while waiting for the lock; that is, once a thread has expressed interest in acquiring the lock, there exists some finite number $N$ such that other threads can perform no more than $N$ subsequent operations on the lock before the first thread acquires it.
		To ensure bounded waiting, we write programs of the following form, in which \texttt{thread1} indicates when it has ``expressed interest in the lock'', and \texttt{thread2} counts the number of times it acquires the lock since then:
	\begin{center} \small
	\begin{tabular}{ll}
	\multicolumn{2}{c}{\texttt{thread1\_waiting = false;}} \\
	\multicolumn{2}{c}{\texttt{thread2\_iters = 0;~~~~~~~}} \\
	\multicolumn{2}{c}{\texttt{fork(thread1, thread2);~~}} \\
	& \\
	\texttt{thread1() \{} & \texttt{thread2() \{} \\
	\texttt{~~~~while (true) \{} & \texttt{~~~~while (true) \{} \\
	\texttt{~~~~~~~~\em lock\_express\_interest\_step();\qquad} & \texttt{~~~~~~~~\em lock\_sequence();} \\
	\texttt{~~~~~~~~thread1\_waiting = true;} & \texttt{~~~~~~~~thread2\_iters++;} \\
	\texttt{~~~~~~~~thread2\_iters = 0;} & \texttt{~~~~~~~~\em unlock\_sequence();} \\
     \texttt{~~~~~~~~\em lock\_acquire\_step();} & \texttt{~~~~\}} \\
	 \texttt{~~~~~~~~thread1\_waiting = false;} & \texttt{\}} \\
	\texttt{~~~~~~~~\em unlock\_sequence();} & \\
	\texttt{~~~~\}} & \\
	\texttt{\}} & \\
	\end{tabular}
	\end{center}
		We then set the goal state to \texttt{thread1\_waiting \&\& thread2\_iters == $N$} for some fixed $N$. If a complete planner cannot plan for this goal, the algorithm satisfies bounded waiting for $N$. If a planner does find a plan for arbitrarily large $N$, bounded waiting is not satisfied.

		This setup also assumes that the lock sequence can be subdivided into the two subsequences denoted above. If there exists no such split, we say that the algorithm does not provide bounded waiting.\footnote{
		This nonexistence can be verified by enumerating all possible instructions in the lock sequence before which to put the new steps. In our test suite, we are not exhaustive about this, though it is possible.}

\paragraph{Progress}
An algorithm that provides progress guarantees that for all $N$, with $N$ operations on the lock, there exists some $M$ number of instructions such that no execution trace longer than $M$ achieves fewer than $N$ operations. (In other words, all operations complete in finite time; there can be no livelocks or deadlocks on a single lock.)

		We could not devise a way to verify progress using the planning technology we had at hand. However, we imagine a hypothetical planner which could perform analysis on the state-space graph to provide this. In planning terminology, progress is stated as follows:
		\begin{itemize}
			\item For all reachable states $S$, there exists a plan from $S$ that acquires+releases the lock an additional time (no deadlock), and
			\item Given some $N$, there exists a finite $M$ such that all plans of length $M$ acquire+release the lock $N$ times (no livelock).
		\end{itemize}

\subsubsection{Results}

Using test cases following the above schemes, we verified or generated counterexamples for mutual exclusion and bounded waiting for four well-known synchronisation algorithms.\footnote{
Pseudocode sketches of each of these algorithms can be found on the wikipedia articles of the same name.}

\begin{itemize}
	\item {\bf Dekker's algorithm.} Demonstrated in the test cases that start with \texttt{prob4}. We verify mutual exclusion and refute bounded waiting.
	\item {\bf Peterson's algorithm.} Demonstrated in the test cases that start with \texttt{prob7}. We verify mutual exclusion and bounded waiting.
	\item {\bf Spinlock (atomic-exchange).} Demonstrated in the test cases that start with \texttt{prob8}. We verify mutual exclusion and refute bounded waiting.
	\item {\bf Lamport's bakery algorithm (atomic-add).} Demonstrated in the test cases that start with \texttt{prob9}. We verify mutual exclusion and bounded waiting.
\end{itemize}

We also include a broken algorithm, \texttt{prob51-broken-lock.pddl}, for which we can even refute mutual exclusion (i.e., the planner can plan for \texttt{num\_in\_section == 2}).

\subsection{Compiler}

To help test our shared memory concurrency domain and to demonstrate
its generality, we built a compiler for a very simple imperative
language with standard structured programming constructs. The compiler
is fairly simple, but makes it vastly nicer to create problems for our
imperative concurrency machine domain.
Appendix~\ref{sec:appendix} includes a motivating example for the compiler:
in Figure
\ref{fig:dekker-code}, we show part of the source program we use for
testing Dekker's algorithm, and in Figure \ref{fig:dekker-asm} we show the
corresponding output problem for our domain. Clearly, the source is
much easier to work with.

The compiler can be invoked with a description of the goal state of
the output program. The goal state can include the program having run
to completion and values for global variables.


\section{Conclusion and Future Work}

In this project, we explored the correspondence between planning and program verification. Planning domains can be used to express an abstract machine's operational semantics, on top of which programs to be verified can be encoded as planning problems. Then goal statements correspond to a program's specification, and output plans to execution traces. We found that planners were unable to execute (and hence, verify) even the simplest possible lambda or combinator terms, due to the highly dynamic, runtime-data-driven nature of the planning problem. We found imperative abstract machines more workable, and verified several synchronisation algorithms.

This approach to formal concurrency verification, whether performed with planners or more conventional model checkers, may be applicable in the growing world of multiprocessor concurrency. Architectures such as ARM (common in cell phones) with weakly-consistent memory models are becoming more popular, and provide opportunities for new, difficult to reason about concurrency routines. Encoding multiprocessor memory semantics as abstract machines would be a logical progression of the work we presented here.

In this project, we used the MIPS (both MIPS-BDD and MIPS-XXL) and especially SGPlan planners. Our codebase, comprising the functional and imperative domains and problems, and the threads compiler, are available at \url{https://github.com/cangiuli/planning}.

\bibliography{citations}{}
\bibliographystyle{alpha}

\newpage

\appendix
\section{Code examples}
\label{sec:appendix}

Here we demonstrate the input format (C-like) and output format (PDDL) of our threads compiler with an example implementation of Dekker's algorithm.

\begin{figure}[h]
\begin{center}
\begin{verbatim}
int flag0 = 0; int flag1 = 0; int turn = 0;

int num_in_section = 0;
int thread1_waiting = 0; int thread2_iters = 0;

thread0() {
    while (1) {
        flag0 = 1;
        thread2_iters = 0; thread1_waiting = 1;

        while (flag1) {
            if (turn) { /* turn != 0 */
                flag0 = 0;
                while (turn) { /* turn != 0 */
                    /* busy wait */
                }
                flag0 = 1;
            }
        }

        /* critical section */
        thread1_waiting = 0;
        num_in_section++;
        num_in_section--;

        turn = 1;
        flag0 = 0;
    }
}

thread1() { /* ELIDED */ }

main() {
    fork(thread0, thread1);
}
\end{verbatim}
\end{center}
\caption{Dekker's algorithm in our simple language}
\label{fig:dekker-code}
\end{figure}

\begin{figure}[h]
\begin{center}
\small
\begin{verbatim}
(define (problem dekker-loop)
    (:domain threads)
    (:objects
        n0 n1 n2 n3 n4 n5 n6 - number
        out - label
        flag0 flag1 turn num_in_section thread1_waiting thread2_iters - label

        thread00 thread01 thread02 thread03 thread04 thread05 thread06 thread07
        thread08 thread09 thread010 thread011 thread012 thread013
        thread10 thread11 thread12 thread13 thread14 thread15 thread16 thread17
        thread18 thread19 thread110 thread111
        main0 main1
        - label
    )
    (:init
        (succ n0 n1) (succ n1 n2) (succ n2 n3)
        (succ n3 n4) (succ n4 n5) (succ n5 n6)
        ; .data
        (value flag0 n0)
        (value flag1 n0)
        (value turn n0)
        (value num_in_section n0)
        (value thread1_waiting n0)
        (value thread2_iters n0)

        ; .text
        ; thread0
        (set thread00 thread01 flag0 n1)
        (set thread01 thread02 thread2_iters n0)
        (set thread02 thread03 thread1_waiting n1)
        (branch thread03 flag1 thread04 thread08)
        (branch thread04 turn thread05 thread03)
        (set thread05 thread06 flag0 n0)
        (branch thread06 turn thread06 thread07)
        (set thread07 thread03 flag0 n1)
        (set thread08 thread09 thread1_waiting n0)
        (incr thread09 thread010 num_in_section)
        (decr thread010 thread011 num_in_section)
        (set thread011 thread012 turn n1)
        (set thread012 thread00 flag0 n0)
        (exit thread013)

        ; thread1
        ; ELIDED
        ; main
        (fork main0 main1 thread00 thread10)
        (exit main1)

        (eval main0 out)
    )
    (:goal (and
            ; INSERT GOALS HERE
        )
    )
)
\end{verbatim}
\end{center}
\caption{The corresponding generated problem for Dekker's algorithm}
\label{fig:dekker-asm}
\end{figure}



\end{document}

The report should be 4-6 pages and include descriptions of (1) the problem,
(2) your approach, and (3) your results.  Please highlight the planning aspects
and other course concepts in all of the sections.  You may want to include some
references to related work in the background for your problem description or in
the evaluation of your results, but you do not need to include a separate
related work survey.

