% Functional and logic programming

\subsection{Plus domain}

Logic programming (as in, e.g., Prolog or Twelf) is a style of declarative
programming in which computations are expressed as predicates which hold exactly
when the conjunction of other predicates hold; executing a program amounts to
searching for a proof of the desired predicate.

A simple example is addition of natural numbers. We represent natural numbers as
either zero or the successor of a natural number ($\texttt{z}, \texttt{s(z)},
\texttt{s(s(z))}$, etc.). Addition is expressed as a predicate
\texttt{plus(M,N,P)}, which holds exactly when $M+N=P$, and could be implemented
in Prolog as:
\begin{verbatim}
plus(z,N,N).
plus(s(M),N,s(P)) :- plus(M,N,P).
\end{verbatim}
The first line says that $\texttt{z}+N=N$, for any $N$, and the second line says
that $\texttt{s(}M\texttt{)}+N=\texttt{s(}P\texttt{)}$ so long as $M+N=P$. This
is remarkably similar to a planning problem, except that Prolog natively
supports producing and consuming nested data like \texttt{s(s(z))} and
instantiates rules via unification of this nested data.

To represent nested data types in PDDL, we were forced to represent them
explicitly via pointer graphs of constructors. \texttt{z} is represented by any
pointer object $\texttt{p}n$ accompanied by the fact \texttt{(Z p$n$)};
similarly, \texttt{s(z)} is represented by an object $\texttt{p}n$ accompanied
by the facts \texttt{(S p$n$ p$m$)} and \texttt{(Z p$m$)}. Thus the first rule
for addition can simply be encoded as an action with precondition \texttt{(Z
?n1)} and effect \texttt{(Plus ?n1 ?n2 ?n2)}.

The second rule, however, requires creating a new \texttt{S} constructor to wrap
around the $P$ result of the recursive call. It is essential that this
constructor is ``allocated'' at a new pointer, to avoid conflicting with any
other constructor in existence. So we keep a fact \texttt{(ptr-next p$n$)} to
store the next currently unused pointer, and increment $n$ every time we use
this pointer (via a set of hardcoded facts \texttt{(ptr-succ p$n$ p$(n+1)$)}).
The implementation looks like this:
{\small
\begin{verbatim}
  (:action Plus-S
    :parameters (?n1 ?n2 ?n1p ?fresh ?freshp ?ansp - ptr)
    :precondition (and
      (Plus ?n1p ?n2 ?ansp)
      (S ?n1 ?n1p)
      (ptr-next ?fresh)
      (ptr-succ ?fresh ?freshp))
    :effect (and
      (not (Plus ?n1p ?n2 ?ansp))
      (not (ptr-next ?fresh))
      (ptr-next ?freshp)
      (Plus ?n1 ?n2 ?fresh)
      (S ?fresh ?ansp)))
\end{verbatim}}
If we strip away all the pointer manipulation, this simply states that if
\texttt{(Plus ?n1p ?n2 ?ansp)}, then \texttt{(Plus (S ?n1p) ?n2 (S ?ansp))}.

Our results here were mixed. SGPlan was immediately able to solve problems like
$2+2$ and $4+2$, but as soon as we provisioned for pointers beyond \texttt{p14}
(with \texttt{ptr-succ} facts), SGPlan would mysteriously exit silently during
the preprocessing phase, regardless of the goal. We couldn't diagnose this
error, but suspect that the planner ran out of memory while computing
constraints between all the \texttt{ptr-succ} facts. As a result, sadly, SGPlan
cannot compute $n+m$ for $2n+m>12$, since this would require over $15$ pointers.

\subsection{Lambda calculus domain}

Although the \texttt{plus} domain above quickly exhausted the planner's
resources, we were at least able to translate a traditional logic program into
PDDL, so we next attempted to implement the lambda calculus in a similar logic
programming fashion. The workhorse of lambda calculus is substitution---when a
function is applied to an argument, we substitute this argument for all
occurrences of the formal parameter in the body of the function. (As in the
\texttt{plus} domain, we represent lambda terms by pointer graphs.)

We attempted to implement substitution by a predicate \texttt{(subst e1 x e2 e)}
which represents the fact that \texttt{e1} substituted for \texttt{x} in
\texttt{e2} is \texttt{e}, and added these rules to an implementation of natural
numbers.

Unfortunately, both MIPS and SGPlan throw an out-of-memory error while
preprocessing these domains, even when the problem is simply to evaluate the
number \texttt{(S Z)}, which evaluates to itself. Given the limited ability of
the planners to perform addition, it is not surprising that substitution, a
complex pointer manipulation problem, was not preprocessable whatsoever.

\subsection{SKI combinator domain}

Our final attempt at logic programming was to encode the SKI combinator
calculus, a formulation of lambda calculus which has no variables whatsoever,
but only three constants $S,K,I$ (and applications thereof) satisfying
particular reduction rules. This has the benefit of sidestepping substitution
entirely, and is the simplest Turing-complete functional language.

Here, we implemented an abstract stack machine following the methodology of
substructural operational semantics (SSOS), a new technique of efficiently
representing abstract machines as transition systems in ordered logic. In this
context, we can think of ordered logic as a planning framework (based on linear
logic%
\footnote{The similarity between linear logic proof search and classical
planning has been observed multiple times in the past, as in \cite{Dixon}.}%
) in which facts are stored in a sequence, and operators consume and
replace sets of contiguous facts in this sequence. This sequencing allows us to
represent stack frames without explicitly keeping track of their order.



Of course, planners also do not support ordered logic, so we encode the stack
frames as a linked list of facts, each of which takes a number of pointers to
combinator expressions, as in the previous domains. The PDDL implementation has
13 operators; we evaluate a combinator expression $E$ by starting with the fact
\texttt{(eval p0 $E$ pn)} (where p0 and pn are the stack frame pointers) and
planning for \texttt{(retn p0 $E$ pn)}, a fact whose existence indicates that
the stack is empty except for that return frame.

Unsurprisingly, SGPlan also runs out of memory while planning on this domain,
even for the simplest possible combinator term, $I$ (the identity combinator).
MIPS attempts to fully instantiate the domain, and cannot even complete
preprocessing. (As a side note, we had to recompile SGPlan to load this domain,
because the \texttt{Retn-S2} instruction has 11 parameters, above the default
maximum of 9.)

\subsection{Analysis}

We were successfully able to encode a variety of logic programs as classical
planning domains, ranging from addition in Prolog to an SSOS encoding of
combinator calculus in ordered logic. This supports our original hypothesis that
logic programming can be viewed as a particular kind of planning problem, and
can be encoded (albeit awkwardly) in PDDL. The biggest hurdle here is that logic
programming is fundamentally based on unification of nested data, and PDDL
requires that we instead express this data using pointers.

The difficulty is that planners attempt to statically analyze the domain and
problem, computing mutual exclusion, subdividing the problem, etc., but these
domains simply cannot be handled in this fashion, because (1) the operators have
many parameters all of the same type, due to the pointer conversion, and (2)
there is precisely one plan to evaluate any addition/lambda term/combinator
term, and each operator in this plan is uniquely determined by the structure of,
respectively, the first summand and the control stack. We tried to alleviate (1)
by using the \texttt{:typing} extension of PDDL to distinguish between, e.g.,
stack frame pointers and expression pointers, to limit the number of
instantiations, but the planners still ran out of memory.

The deterministic nature of the operational semantics applies only to
``well-formed'' initial states, and is a non-trivial fact. It is in fact very
easy to find this unique plan by starting at the initial state, determining the
unique operator and instantiation which can be applied at each state, and
repeating until the goal is satisfied; the planners' preprocessing consumes all
available memory but does not actually simplify the problem whatsoever. 


